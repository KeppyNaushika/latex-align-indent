\documentclass{article}
\usepackage{tikz}
\usepackage{pgf}

\begin{document}

  % Problem case 1: foreach loop - unindented
  \foreach \x in {-15,-14,...,15} {
  \draw[dash pattern=on 0 off .25mm, line cap=round] (\x,-15.25) -- (\x,15.25);
  }\

  % Problem case 2: Complex newcommand - unindented
  \newcommand{\drawInCoordinatePlane}[7]{
  % #1: fx (関数の式、例: \x+2)
  % #2: x_min
  % #3: x_max
  % #4: y_min
  % #5: y_max
  % #6: node_option (例: above right)
  % #7: text (例: $y=x+2$)

  % clippingして関数を描画
  \begin{scope}
    \clip (#2,#4) rectangle (#3,#5);
    \draw[thick, domain=#2:#3] plot (\x, {#1});
  \end{scope}

  % x=#3での関数値を直接計算
  \pgfmathsetmacro{\xmax}{#3}
  \def\x{#3}
  \pgfmathsetmacro{\yatxmax}{#1}

  % 範囲チェックとnode位置決定
  \pgfmathparse{\yatxmax > #5 ? 1 : (\yatxmax < #4 ? -1 : 0)}
  \pgfmathtruncatemacro{\rangecheck}{\pgfmathresult}

  \ifnum\rangecheck=0
  % 範囲内の場合
  \node[#6] at (#3,\yatxmax) {#7};
  \else
  % 範囲外の場合、f(x) = y_target となるx座標を計算
  \pgfmathsetmacro{\ytarget}{\rangecheck > 0 ? #5 : #4}

  % 一次関数 f(x) = ax + b から係数を抽出
  % #1の形式に応じて係数を求める
  \pgfmathsetmacro{\slope}{#1}  % まずx=1で計算
  \def\x{1}
  \pgfmathsetmacro{\atone}{#1}  % f(1)を計算
  \def\x{0}
  \pgfmathsetmacro{\atzero}{#1} % f(0) = bを計算
  \pgfmathsetmacro{\slope}{\atone - \atzero}  % a = f(1) - f(0)
  \pgfmathsetmacro{\intercept}{\atzero}       % b = f(0)

  % f(x) = y_target となるx座標を計算: x = (y_target - b) / a
  \pgfmathsetmacro{\xintersect}{(\ytarget - \intercept) / \slope}

  \node[#6] at (\xintersect,\ytarget) {#7};
  \fi
  }\

\end{document}